


Let us begin with $\deriv{}{x_2}{2}u(r_{12},\mur{1})$
\begin{equation}
 \begin{aligned}
  \deriv{}{x_2}{2}u(r_{12},\mur{1}) & = - \deriv{}{x_2}{} \bigg( \frac{1 - \text{erf}\big(\mu(\bri{1}) r_{12} \big)}{2 r_{12}}\big( x_1 - x_2) \bigg) \\
                                    & =  \frac{1 - \text{erf}\big(\mu(\bri{1}) r_{12} \big)}{2 r_{12}} \\ & - (x_1 - x_2) \deriv{}{x_2}{} \frac{1 - \text{erf}\big(\mu(\bri{1}) r_{12} \big)}{2 r_{12}}
 \end{aligned}
\end{equation}
Similarly 
\begin{equation}
 \begin{aligned}
  \deriv{}{x_1}{}u(r_{12},\mur{1}) & = \deriv{}{x_1}{}\bigg( \frac{1 - \text{erf}\big(\mu(\bri{1}) r_{12} \big)}{2 r_{12}}\big( x_1 - x_2) \bigg) \\
                                   & = \frac{1 - \text{erf}\big(\mu(\bri{1}) r_{12} \big)}{2 r_{12}} \\ 
                                   & + (x_1 - x_2) \deriv{}{x_1}{} \frac{1 - \text{erf}\big(\mu(\bri{1}) r_{12} \big)}{2 r_{12}} \\ 
                                   & + \frac{1 - \text{erf}\big(\mu(\bri{1}) r_{12} \big)}{2 r_{12}}\big( x_1 - x_2) \deriv{}{x_1}{} \mur{1}. 
 \end{aligned}
\end{equation}
Also, as 
\begin{equation}
 \label{eq:der_erfc_r12}
 \begin{aligned}
 &\deriv{}{x_1}{} \frac{1 - \text{erf}\big(\mu(\bri{1}) r_{12} \big)}{2 r_{12}}  = - \deriv{}{x_2}{} \frac{1 - \text{erf}\big(\mu(\bri{1}) r_{12} \big)}{2 r_{12}} \\
 & =  -\frac{(x_1 - x_2)}{\big( r_{12} \big)^2}\bigg(\frac{1 - \text{erf}\big( \mur{1} r_{12} \big)}{2 r_{12}}  + \frac{\mur{1}}{\sqrt{\pi}} e^{-\big(\mur{1} r_{12} \big)^2}      \bigg), 
 \end{aligned}
\end{equation}
one has that 
\begin{equation}
 \begin{aligned}
  \deriv{}{x_2}{2}u(r_{12},\mur{1}) & = \frac{1 - \text{erf}\big(\mu(\bri{1}) r_{12} \big)}{2 r_{12}} \\
                                    & + (x_1 - x_2) \deriv{}{x_1}{} \frac{1 - \text{erf}\big(\mu(\bri{1}) r_{12} \big)}{2 r_{12}}. 
 \end{aligned}
\end{equation}
Therefore, 
\begin{equation}
 \begin{aligned}
 & \deriv{}{x_2}{2}u(r_{12},\mur{1}) + \deriv{}{x_2}{2}u(r_{12},\mur{1})  = 2 \times \frac{1 - \text{erf}\big(\mu(\bri{1}) r_{12} \big)}{2 r_{12}} \\
                                    & + 2 \times (x_1 - x_2) \deriv{}{x_1}{} \frac{1 - \text{erf}\big(\mu(\bri{1}) r_{12} \big)}{2 r_{12}} \\ 
& + \frac{1 - \text{erf}\big(\mu(\bri{1}) r_{12} \big)}{2 r_{12}}\big( x_1 - x_2) \deriv{}{x_1}{} \mur{1}, 
 \end{aligned}
\end{equation}
which, according to Eq. \eqref{eq:der_erfc_r12}, becomes  
\begin{equation}
 \begin{aligned}
 & \deriv{}{x_2}{2}u(r_{12},\mur{1}) + \deriv{}{x_2}{2}u(r_{12},\mur{1}) = \frac{1 - \text{erf}\big(\mu(\bri{1}) r_{12} \big)}{r_{12}} \\ 
 & - 2 \times \frac{(x_1 - x_2)^2}{\big( r_{12} \big)^2}\bigg(\frac{1 - \text{erf}\big( \mur{1} r_{12} \big)}{2 r_{12}}  + \frac{\mur{1}}{\sqrt{\pi}} e^{-\big(\mur{1} r_{12} \big)^2}      \bigg) \\
& + \frac{1 - \text{erf}\big(\mu(\bri{1}) r_{12} \big)}{2 r_{12}}\big( x_1 - x_2) \deriv{}{x_1}{} \mur{1}. 
 \end{aligned}
\end{equation}
Therefore, 
\begin{equation}
 \begin{aligned}
& \nabla_1 \cdot \nabla_1 u(r_{12},\mur{1}) +  \nabla_2 \cdot \nabla_2 u(r_{12},\mur{1}) = 
 3 \frac{1 - \text{erf}\big(\mu(\bri{1}) r_{12} \big)}{r_{12}} \\
& - 2 \bigg(\frac{1 - \text{erf}\big( \mur{1} r_{12} \big)}{2 r_{12}}  +                   \frac{\mur{1}}{\sqrt{\pi}} e^{-\big(\mur{1} r_{12} \big)^2}      \bigg) \\
& + \frac{1 - \text{erf}\big(\mu(\bri{1}) r_{12} \big)}{2 r_{12}}\big( \br_1 - \br_2) \cdot \nabla_1 \mur{1}. 
 \end{aligned}
\end{equation}
or equivalently 
\begin{equation}
 \begin{aligned}
& \nabla_1 \cdot \nabla_1 u(r_{12},\mur{1}) +  \nabla_2 \cdot \nabla_2 u(r_{12},\mur{1})  \\
 =& +2 \bigg( \frac{1 - \text{erf}\big(\mu(\bri{1}) r_{12} \big)}{r_{12}} -  \frac{\mur{1}}{\sqrt{\pi}} e^{-\big(\mur{1} r_{12} \big)^2} \bigg) \\
& + \frac{1 - \text{erf}\big(\mu(\bri{1}) r_{12} \big)}{2 r_{12}}\big( \br_1 - \br_2) \cdot \nabla_1 \mur{1}. 
 \end{aligned}
\end{equation}

Let us begin by the "easy" term $\nabla_2 \cdot \nabla_2 u(r_{12};\mur{1})$
\begin{equation}
 \begin{aligned}
 \label{eq:lapl_u2}
 \nabla_2 \cdot \nabla_2 u(r_{12};\mur{1}) =& -\bigg( \deriv{}{x_2}{} {\bf e_{x_2}}+ \deriv{}{y_2}{} {\bf e_{y_2}}+ \deriv{}{y_2}{} {\bf e_{z_2}} \bigg) \\ & \cdot \bigg(\big(\br_1 - \br_2 \big)\frac{1 - \text{erf}(\mur{1} r_{12})}{2 r_{12}}\bigg) \\
          =&-\bigg( L_2^x(\br_1, r_{12}) + L_2^y(\br_1, r_{12})  + L_2^z(\br_1, r_{12})\bigg)
 \end{aligned}
\end{equation}
with 
\begin{equation}
 L_2^x(\br_1, r_{12}) = \deriv{}{x_2}{} \big(x_1 - x_2 \big)\frac{1 - \text{erf}(\mur{1} r_{12})}{2 r_{12}}.
\end{equation}
Notice the minus sign in Eq. \eqref{eq:lapl_u2}. 
Similarly, the term $\nabla_1 \cdot \nabla_1 u(r_{12};\mur{1})$ can be written into a sum of two terms 
\begin{equation}
 \begin{aligned}
 &\nabla_1 \cdot \nabla_1 u(r_{12};\mur{1}) =  \bigg( \deriv{}{x_1}{} {\bf e_{x_1}}+ \deriv{}{y_1}{} {\bf e_{y_1}}+ \deriv{}{y_1}{} {\bf e_{z_2}} \bigg) \\ & \cdot \bigg( {\bf {\gamma}}(\br_1, r_{12}) - \nabla_2 u(r_{12};\mur{1}) \bigg) \\
 \end{aligned}
\end{equation}
Let us focus first on the second term 
\begin{equation}
 \begin{aligned}
 &\bigg( \deriv{}{x_1}{} {\bf e_{x_1}}+ \deriv{}{y_1}{} {\bf e_{y_1}}+ \deriv{}{y_1}{} {\bf e_{z_2}} \bigg)  \cdot \bigg( -\nabla_2 u(r_{12};\mur{1}) \bigg) \\
         = &  -\big(L_1^x(\br_1, r_{12}) + L_1^y(\br_1, r_{12})  + L_1^z(\br_1, r_{12}) \big) ,
 \end{aligned}
\end{equation}
with 
\begin{equation}
 \begin{aligned}
 L_1^x(\br_1, r_{12})& =  \deriv{}{x_1}{} \deriv{}{x_2}{} u(r_{12};\mur{1}) \\
                     & =  \deriv{}{x_1}{} \big(x_1 - x_2 \big)\frac{1 - \text{erf}(\mur{1} r_{12})}{2 r_{12}} \\ 
                     & + \big(x_1 - x_2 \big)\frac{1 - \text{erf}(\mur{1} r_{12})}{2 r_{12}} \deriv{}{x_1}{} \mur{1}.
 \end{aligned}
\end{equation}
Therefore, let us compute 
First let us notice that 
\begin{equation}
 \begin{aligned}
 \label{eq:d2_x1_2}
& \deriv{}{x_1}{}\bigg( (x_1 - x_2) \frac{1 - \text{erf}(\mur{1} r_{12})}{2 r_{12}} \bigg) \\
&=  \bigg( \deriv{}{x_1}{} \frac{1 - \text{erf}(\mur{1} r_{12})}{2 r_{12}}\bigg) (x_1 - x_2 ) +  \frac{1 - \text{erf}(\mur{1} r_{12})}{2 r_{12}}  \\
& = -\frac{(x_1 - x_2)^2}{\big( r_{12} \big)^2}\bigg(\frac{1 - \text{erf}\big( \mur{1} r_{12} \big)}{2 r_{12}}  + \frac{\mur{1}}{\sqrt{\pi}} e^{-\big(\mur{1} r_{12} \big)^2}   \bigg) +  \frac{1 - \text{erf}(\mur{1} r_{12})}{2 r_{12}} \\
& = \frac{1 - \text{erf}(\mur{1} r_{12})}{2 r_{12}} \bigg( 1 - \frac{(x_1 - x_2)^2}{\big( r_{12} \big)^2} \bigg) 
- \frac{(x_1 - x_2)^2}{\big( r_{12} \big)^2}\frac{\mur{1}}{\sqrt{\pi}} e^{-\big(\mur{1} r_{12} \big)^2},
 \end{aligned}
\end{equation}


\subsection{Computation of the integrals of the Laplacian }
The computation of the analytical form of the Laplacian is rather tedious, so therefore we propose to compute these integrals using the integration by part using 
\begin{equation}
 \begin{aligned}
&  \matelem{kl}{\nabla^2_1 u(r_{12};\mur{1})}{ij} \\ 
= & \int \text{d}\br_{1} \text{d}\br_{2}   \phi_l(\br_2) \phi_j(\br_2) \nabla^2_1 u(r_{12};\mur{1}) \phi_i(\br_1) \phi_k(\br_1) \\
= & -\int \text{d}\br_{1} \text{d}\br_{2}  \phi_l(\br_2) \phi_j(\br_2) \nabla_1 u(r_{12};\mur{1}) \\ & \cdot \big[ \phi_i(\br_1) \nabla_1 \phi_k(\br_1) +  \phi_k(\br_1) \nabla_1 \phi_i(\br_1)\big].
 \end{aligned}
\end{equation}
Therefore one has 
\begin{equation}
 \label{eq:lapl_1}
 \begin{aligned}
&  \matelem{kl}{\nabla^2_1 u(r_{12};\mur{1})}{ij} \\ 
= & -\int \text{d}\br_{1} \text{d}\br_{2}  \phi_l(\br_2) \phi_j(\br_2) \nabla_1 u(r_{12};\mur{1}) \\ & \cdot \big[ \phi_i(\br_1) \nabla_1 \phi_k(\br_1) +  \phi_k(\br_1) \nabla_1 \phi_i(\br_1)\big].
 \end{aligned}
\end{equation}
Then one has also to compute the same quantity for $\br_2$:
\begin{equation}
 \label{eq:lapl_2}
 \begin{aligned}
&  \matelem{kl}{\nabla^2_2 u(r_{12};\mur{1})}{ij} \\ 
= & -\int \text{d}\br_{1} \text{d}\br_{2} \phi_i(\br_1) \phi_k(\br_1) \nabla_2 u(r_{12};\mur{1}) \\ & \cdot \big[\phi_l(\br_2)  \nabla_2  \phi_j(\br_2) + \phi_j(\br_2)  \nabla_2  \phi_l(\br_2)   \big].
 \end{aligned}
\end{equation}

\subsection{Computation of the integrals of the non hermitian term}
The two integrals involved in the non hermitian term are the following: 
\begin{equation}
 \label{eq:nh_1}
 \begin{aligned}
&   \matelem{kl}{ \nabla_1 u(\bri{1},\bri{2}) \cdot \nabla_2 }{ij} \\ 
=& \int \text{d}\br_{1} \text{d}\br_{2} \phi_k(\br_1) \phi_l(\br_2) \phi_i(\br_1) \nabla_1 u(\bri{1},\bri{2}) \cdot \nabla_2 \phi_j(\br_2), 
 \end{aligned}
\end{equation}
and 
\begin{equation}
 \label{eq:nh_2}
 \begin{aligned}
&   \matelem{kl}{ \nabla_2 u(\bri{1},\bri{2}) \cdot \nabla_1 }{ij} \\ 
=& \int \text{d}\br_{1} \text{d}\br_{2} \phi_k(\br_1) \phi_l(\br_2) \phi_j(\br_2) \nabla_2 u(\bri{1},\bri{2}) \cdot \nabla_1 \phi_i(\br_1). 
 \end{aligned}
\end{equation}

\subsection{Integrals to be computed for the Laplacian and non hermitian term}
\subsubsection{Sum of all terms}
Regarding the two-body part of the effective TC operator $\hat{K}(\bri{1},\bri{2}) ) $, one can notice that there are some terms which cancel each other between Eq. \eqref{eq:lapl_1}, \eqref{eq:lapl_2} and \eqref{eq:nh_1}, \eqref{eq:nh_2}. 
From Eqs. \eqref{eq:grad_mur_1} and \eqref{eq_grad_mur_r2}, one can write that
\begin{equation}
 \nabla_1 u(r_{12};\mur{1})  = - \nabla_2 u(r_{12};\mur{1}) + {\bf {\gamma}}(\br_1, r_{12}), 
\end{equation}
with 
\begin{equation}
 {\bf {\gamma}}(\br_1, r_{12}) = \frac{e^{-(\mu(\bri{1}) r_{12})^2}}{2 \sqrt{\pi} \mu(\bri{1})^2} \nabla_1{\mur{1}}
\end{equation}
Taking the chemist notation that
\begin{equation}
 \big(jl|f(\br_1,\br_2) \big) = \int \text{d}\br_{1} \text{d}\br_{2} \phi_l(\br_2) \phi_j(\br_2) f(\br_1,\br_2) 
\end{equation}
\begin{equation}
 \begin{aligned}
& \matelem{kl}{\frac{1}{2}\bigg( \nabla_1^2 u(\bri{1},\bri{2}) + \nabla_2^2u(\bri{1},\bri{2}) + \nabla_1 u(\bri{1},\bri{2}) \cdot \nabla_2 + \nabla_2 u(\bri{1},\bri{2}) \cdot \nabla_1 \bigg)}{ij} \\
 =& -\big(jl| \nabla_1 u(r_{12};\mur{1}) \big[ \phi_k(\br_1) \nabla_1 \phi_i(\br_1) + \phi_i(\br_1)\nabla_1 \phi_k(\br_1)\big] \big) \\
 &  +\big(jl| \nabla_2 u(r_{12};\mur{1}  \,\, \,\,      \phi_k(\br_1) \nabla_1 \phi_i(\br_1) \\
 &  -\big(ik| \nabla_2 u(r_{12};\mur{1}) \big[ \phi_l(\br_2) \nabla_2 \phi_j(\br_2) + \phi_j(\br_2)\nabla_2 \phi_l(\br_2)\big] \big) \\
 &  +\big(ik| \nabla_1 u(r_{12};\mur{1}) \,\, \,\,      \phi_l(\br_2) \nabla_2 \phi_j(\br_2) \\
 =& -2     \big(ik|\nabla_2u(r_{12};\mur{1}) \phi_l(\br_2) \nabla_2 \phi_j(\br_2)\big) \\
  & -\,\,\,\big(ik|\nabla_2u(r_{12};\mur{1}) \phi_j(\br_2)\nabla_2 \phi_l(\br_2) \big) \\
  & +       \big(ik|{\bf {\gamma}}(\br_1, r_{12}) \big[ \phi_l(\br_2)\nabla_2 \phi_j(\br_2) + \phi_j(\br_2)\nabla_2 \phi_l(\br_2) \big] \big) \\
  & +2      \big(jl|\nabla_2 u(r_{12};\mur{1}) \phi_k(\br_1) \nabla_1 \phi_i(\br_1) \big) \\
  & + \,\,\,\big(jl|\nabla_2 u(r_{12};\mur{1}) \phi_i(\br_1) \nabla_1 \phi_k(\br_1) \big) \\
  & - \,\,\,\big(jl|{\bf {\gamma}}(\br_1, r_{12}) \big[ \phi_k(\br_1) \nabla_1 \phi_i(\br_1) + \phi_i(\br_1) \nabla_1 \phi_k(\br_1) \big] \big)
 \end{aligned}
\end{equation}
or in a more compact form 
\begin{equation}
 \label{eq:lapl_nh_0}
 \begin{aligned}
& \matelem{kl}{\frac{1}{2}\bigg( \nabla_1^2 u(\bri{1},\bri{2}) + \nabla_2^2u(\bri{1},\bri{2}) + \nabla_1 u(\bri{1},\bri{2}) \cdot \nabla_2 + \nabla_2 u(\bri{1},\bri{2}) \cdot \nabla_1 \bigg)}{ij} \\
=& -\frac{1}{2} \big(ik|\nabla_2u(r_{12};\mur{1}) \big[ \phi_j(\br_2)\nabla_2 \phi_l(\br_2) + 2 \phi_l(\br_2) \nabla_2 \phi_j(\br_2)\big] \big) \\
 & +\frac{1}{2} \big(jl|\nabla_2u(r_{12};\mur{1}) \big[ \phi_i(\br_1)\nabla_1 \phi_k(\br_1) + 2 \phi_k(\br_1) \nabla_1 \phi_i(\br_1)\big] \big) \\
 & +\frac{1}{2} \big(ik|{\bf {\gamma}}(\br_1, r_{12}) \big[ \phi_l(\br_2)\nabla_2 \phi_j(\br_2) + \phi_j(\br_2)\nabla_2 \phi_l(\br_2) \big] \big) \\
 & -\frac{1}{2} \big(jl|{\bf {\gamma}}(\br_1, r_{12}) \big[ \phi_k(\br_1) \nabla_1 \phi_i(\br_1) + \phi_i(\br_1) \nabla_1 \phi_k(\br_1) \big] \big).
 \end{aligned}
\end{equation}
The integrals in Eq. \eqref{eq:lapl_nh_0} can be obtained from mixed numerical analytical integration: one analytically integrates all dependency on $\br_2$ and numerically integrates the $\br_1$ variable. 

\subsubsection{Integrals involving $\nabla_2u(r_{12};\mur{1})$}
Let us take for instance the first integral of Eq. \eqref{eq:lapl_nh_0}:
\begin{equation}
 \label{eq:lapl_nh_1}
 \begin{aligned}
 &\big(ik|\nabla_2 u(r_{12};\mur{1}) \big[ \phi_j(\br_2)\nabla_2 \phi_l(\br_2) + 2 \phi_l(\br_2) \nabla_2 \phi_j(\br_2)\big] \big) \\
 &= -\int \text{d}\br_{1} \text{d}\br_{2} \frac{1 - \text{erf}(\mur{1} r_{12})}{2 r_{12}} \big(\br_1 - \br_2 \big) \cdot\\
 &   \phi_i(\br_1) \phi_k(\br_1) \big[ \phi_j(\br_2)\nabla_2 \phi_l(\br_2) + 2 \phi_l(\br_2) \nabla_2 \phi_j(\br_2)\big] .
 \end{aligned}
\end{equation}
Such an integral can be decomposed as the sum of the three components of the scalar product 
\begin{equation}
 \label{eq:lapl_nh_2}
 \begin{aligned}
 &\big(ik|\nabla_2 u(r_{12};\mur{1}) \big[ \phi_j(\br_2)\nabla_2 \phi_l(\br_2) + 2 \phi_l(\br_2) \nabla_2 \phi_j(\br_2)\big] \big) \\
&=-\big( I_{ijkl}^x + I_{ijkl}^y + I_{ijkl}^z\big),
 \end{aligned}
\end{equation}
where for instance the $x$ component  of Eq. \eqref{eq:lapl_nh_2} is simply 
\begin{equation}
 \label{eq:lapl_nh_3}
 \begin{aligned}
 &I_{ijkl}^x = \int \text{d}\br_{1} \text{d}\br_{2} \frac{1 - \text{erf}(\mur{1} r_{12})}{2 r_{12}} \big(x_1 - x_2 \big) \\
 &   \phi_i(\br_1) \phi_k(\br_1) \big[ \phi_j(\br_2)\deriv{}{x_2}{} \phi_l(\br_2) + 2 \phi_l(\br_2) \deriv{}{x_2}{} \phi_j(\br_2)\big] .
 \end{aligned}
\end{equation}
By defining the following integrals
\begin{equation}
 \label{eq:u_jl}
 \mathcal{U}_{jl}^{x}(\br) = \int \text{d}\br' \frac{1 - \text{erf}(\mu (\br') |\br - \br'|)}{2 |\br - \br'|} \phi_j(\br') \deriv{}{x'}{} \phi_l(\br'),
\end{equation}
\begin{equation}
 \label{eq:ux_jl}
 \mathcal{U}_{jl}^{xx}(\br) = \int \text{d}\br' x' \frac{1 - \text{erf}(\mu (\br') |\br - \br'|)}{2 |\br - \br'|} \phi_j(\br') \deriv{}{x'}{} \phi_l(\br'),
\end{equation}
then the integral $I_{ijkl}^x $ of Eq. \eqref{eq:lapl_nh_3} becomes 
\begin{equation}
 \label{eq:int_i_x}
 \begin{aligned}
 &I_{ijkl}^x = \int \text{d}\br \phi_i(\br) \phi_k(\br) \bigg[ x \mathcal{U}_{jl}^x(\br) - \mathcal{U}_{jl}^{xx}(\br) + 2 \bigg(x \mathcal{U}_{lj}^x(\br) - \mathcal{U}_{lj}^{xx}(\br) \bigg)\bigg].
 \end{aligned}
\end{equation}
Then, the second term of Eq. \eqref{eq:lapl_nh_0} can be obtained with a similar expression
\begin{equation}
 \begin{aligned}
 & \big(jl|\nabla_2u(r_{12};\mur{1}) \big[ \phi_i(\br_1)\nabla_1 \phi_k(\br_1) + 2 \phi_k(\br_1) \nabla_1 \phi_i(\br_1)\big] \big) \\
 &=-\big( J_{ijkl}^x + J_{ijkl}^y + J_{ijkl}^z \big)
 \end{aligned}
\end{equation}
where 
\begin{equation}
 \begin{aligned}
 \label{eq:int_j_x}
  J_{ijkl}^x = \int \text{d}\br_1 \text{d}\br_2 & \bigg[ \phi_i(\br_1)\deriv{}{x_1}{} \phi_k(\br_1) + 2 \phi_k(\br_1) \deriv{}{x_1}{} \phi_i(\br_1)\bigg] \\  
 & \frac{1 - \text{erf}(\mur{1} r_{12})}{2 r_{12}} \big(x_1 - x_2 \big) \phi_j(\br_2) \phi_l(\br_2).
 \end{aligned}
\end{equation}
Therefore, by defining the following integrals 
\begin{equation}
 \label{eq:u_jl}
 \mathcal{V}_{jl}(\br) = \int \text{d}\br' \frac{1 - \text{erf}(\mu (\br') |\br - \br'|)}{2 |\br - \br'|} \phi_j(\br') \phi_l(\br'),
\end{equation}
\begin{equation}
 \label{eq:ux_jl}
 \mathcal{V}_{jl}^{x}(\br) = \int \text{d}\br' x' \frac{1 - \text{erf}(\mu (\br') |\br - \br'|)}{2 |\br - \br'|} \phi_j(\br') \phi_l(\br'),
\end{equation}
one can re express the integral $J_{ijkl}^x$ in Eq. \eqref{eq:int_j_x} as 
\begin{equation}
 \begin{aligned}
& J_{ijkl}^x = \int \text{d}\br \bigg[ \phi_i(\br)\deriv{}{x}{} \phi_k(\br) + 2 \phi_k(\br) \deriv{}{x}{} \phi_i(\br)\bigg]  \big( x \mathcal{V}_{jl}(\br) -  \mathcal{V}_{jl}^{x}(\br) \big).
 \end{aligned}
\end{equation}

Eventually, the two first lines of the right hand side of Eq. \eqref{eq:lapl_nh_0} can be re written as  
\begin{equation}
 \label{eq:lapl_nh_end}
 \begin{aligned}
 & -\frac{1}{2} \big(ik|\nabla_2u(r_{12};\mur{1}) \big[ \phi_j(\br_2)\nabla_2 \phi_l(\br_2) + 2 \phi_l(\br_2) \nabla_2 \phi_j(\br_2)\big] \big) \\
 & +\frac{1}{2} \big(jl|\nabla_2u(r_{12};\mur{1}) \big[ \phi_i(\br_1)\nabla_1 \phi_k(\br_1) + 2 \phi_k(\br_1) \nabla_1 \phi_i(\br_1)\big] \big) \\
 =&  - \frac{1}{2} \bigg( -\big( I_{ijkl}^x + I_{ijkl}^y + I_{ijkl}^z\big) \bigg) \\
 &  +  \frac{1}{2} \bigg( -\big( J_{ijkl}^x + J_{ijkl}^y + J_{ijkl}^z \big) \bigg) \\
 =& \frac{1}{2} \bigg( I_{ijkl}^x + I_{ijkl}^y + I_{ijkl}^z - J_{ijkl}^x + J_{ijkl}^y + J_{ijkl}^z \bigg). 
 \end{aligned}
\end{equation}

\subsubsection{Integrals involving ${\bf {\gamma}}(\br_1, r_{12})$}
The last types of integrals one needs to compute are those involving ${\bf {\gamma}}(\br_1, r_{12})$ in Eq. \eqref{eq:lapl_nh_0}. 
The first term in ${\bf {\gamma}}(\br_1, r_{12})$ is
\begin{equation}
 \begin{aligned}
 \big(ik|{\bf {\gamma}}(\br_1, r_{12}) \big[ \phi_l(\br_2)\nabla_2 \phi_j(\br_2) + \phi_j(\br_2)\nabla_2 \phi_l(\br_2) \big] \big),
 \end{aligned}
\end{equation}
which can be also decomposed into $x,y,z$ components 
\begin{equation}
 \begin{aligned}
 & \big(ik|{\bf {\gamma}}(\br_1, r_{12}) \big[ \phi_l(\br_2)\nabla_2 \phi_j(\br_2) + \phi_j(\br_2)\nabla_2 \phi_l(\br_2) \big] \big) \\
 & = M_{ijkl}^x  + M_{ijkl}^y  + M_{ijkl}^z 
 \end{aligned}
\end{equation}
where 
\begin{equation}
 \label{eq:int_m_x}
 \begin{aligned}
  M_{ijkl}^x =  & \int \text{d}\br_1 \text{d}\br_2 \frac{e^{-(\mu(\bri{1}) r_{12})^2}}{2 \sqrt{\pi} \mu(\bri{1})^2} \deriv{}{x_1}{} \mur{1} \\
                & \phi_i(\br_1) \phi_k(\br_1) \big[ \phi_l(\br_2)\deriv{}{x_2}{}\phi_j(\br_2) + \phi_j(\br_2)\deriv{}{x_2}{} \phi_l(\br_2)  \big].
 \end{aligned}
\end{equation}
By defining the following integral 
\begin{equation}
 \mathcal{G}^x_{jl}(\br) = \int \text{d}\br' e^{-(\mu(\br) |\br - \br'|)^2} 
                \phi_l(\br')\deriv{}{x}{}\phi_j(\br'), 
\end{equation}
one can rewrite $M_{ijkl}^x $ of Eq. \eqref{eq:int_m_x} as 
\begin{equation}
 M_{ijkl}^x = \int \text{d}\br \phi_i(\br) \phi_k(\br) \frac{1}{2 \sqrt{\pi} \mu(\bri{1})^2}\deriv{}{x_1}{} \mur{1} \big[\mathcal{G}^x_{jl}(\br) + \mathcal{G}^x_{lj}(\br) \big]. 
\end{equation}
Eventually, the second integral of Eq. \eqref{eq:lapl_nh_0} can also be re written as $x,y,z$ components 
\begin{equation}
 \begin{aligned}
 \big(jl|{\bf {\gamma}}(\br_1, r_{12}) \big[ \phi_k(\br_1) \nabla_1 \phi_i(\br_1) + \phi_i(\br_1) \nabla_1 \phi_k(\br_1) \big] \big) 
 & = N_{ijkl}^x + N_{ijkl}^y + N_{ijkl}^z, 
 \end{aligned}
\end{equation}
where $N_{ijkl}^x $ is defined as 
\begin{equation}
 \begin{aligned}
 \label{eq:int_n_x}
  N_{ijkl}^x  = & \int \text{d}\br_1 \text{d}\br_2 \frac{e^{-(\mu(\bri{1}) r_{12})^2}}{2 \sqrt{\pi} \mu(\bri{1})^2}       \deriv{}{x_1}{} \mur{1} \\
& \phi_j(\br_2) \phi_l(\br_2) \big[ \phi_k(\br_1) \deriv{}{x_1}{} \phi_i(\br_1) + \phi_i(\br_1) \deriv{}{x_1}{} \phi_k(\br_1)  \big] \big).
 \end{aligned}
\end{equation}
By defining the following integral 
\begin{equation}
 \mathcal{G}^{jl}(\br) = \int \text{d}\br' e^{-(\mu(\br) |\br - \br'|)^2} 
                \phi_l(\br')\phi_j(\br'), 
\end{equation}
one can rewrite $N_{ijkl}^x $ of Eq. \eqref{eq:int_m_x} as 
\begin{equation}
 \begin{aligned}
  N_{ijkl}^x = & \int \text{d}\br \frac{1}{2 \sqrt{\pi} \mu(\br)^2}\deriv{}{x}{} \mur{} \\ & \mathcal{G}^{jl}(\br) \big[ \phi_k(\br) \deriv{}{x}{}\phi_i(\br) + \phi_i(\br) \deriv{}{x}{} \phi_k(\br)  \big]. 
 \end{aligned}
\end{equation}

