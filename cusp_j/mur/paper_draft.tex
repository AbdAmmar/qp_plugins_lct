
\documentclass[aip,jcp,reprint,noshowkeys,superscriptaddress]{revtex4-1}
\usepackage{graphicx,dcolumn,bm,xcolor,microtype,multirow,amsmath,amssymb,amsfonts,physics,mhchem,xspace,subfigure}

\usepackage[utf8]{inputenc}
\usepackage[T1]{fontenc}
\usepackage{txfonts}

\usepackage[
	colorlinks=true,
    citecolor=blue,
    breaklinks=true
	]{hyperref}
\urlstyle{same}

\definecolor{darkgreen}{HTML}{009900}
\usepackage[normalem]{ulem}
\newcommand{\sphi}[1]{\hat{{\bf S}}_{#1}}
\newcommand{\overlap}[2]{\langle #1 | #2 \rangle}
\newcommand{\matelem}[3]{\langle #1 | #2 | #3 \rangle}
\newcommand{\deriv}[3]{\frac{\partial^{#3} #1}{\partial {#2}^{#3}}}
\newcommand{\bd}[1]{{\bf {#1}}}
\newcommand{\br}[0]{{\bf {r}}}
\newcommand{\bri}[1]{{\bf r}_{#1}}
\newcommand{\bs}[0]{{\bf {s}}}
\newcommand{\dr}[1]{\text{d}{\bf r_{#1}}}
\newcommand{\psiex}[0]{\Psi^{\text{ex}}_0}
\newcommand{\phiex}[0]{\Phi^{\text{ex}}_0}
\newcommand{\phimu}[0]{\Phi^{\text{ex},\mu}_0}
\newcommand{\phimub}[0]{\Phi^{\mathcal{B},\mu}_0}
\newcommand{\xhimub}[0]{X^{\mathcal{B},\mu}_0}
\newcommand{\psimub}[0]{\Psi^{\mathcal{B},\mu}_0}
\newcommand{\phiimub}[0]{\Phi^{\mathcal{B},\mu}_i}
\newcommand{\basis}[0]{\mathcal{B}}
\newcommand{\energyex}[0]{E^{\text{ex}}}
\newcommand{\R}{\mathbb{R}}
\newcommand{\identity}{\mathds{1}}
\newcommand{\mur}[1]{\mu({\bf r_{#1}})}
\newcommand{\muueg}{\mu_{\text{UEG}}}
\newcommand{\muuegav}{\langle \mu_{\text{UEG}}\rangle}
\newcommand{\muav}{\langle \mu\rangle}
\newcommand{\mursc}{ \mu_{r_{s,c}}}
\newcommand{\murscav}{\langle \mu_{r_{s,c}}\rangle}
\newcommand{\mursclda}{\langle \mu_{r_{s,c}^{\text{UEG}}}\rangle}



\begin{document}	

\title{A new form of transcorrelated Hamiltonian inspired by range-separated DFT: II }

\author{Emmanuel Giner}
\email{emmanuel.giner@lct.jussieu.fr}

\begin{abstract}

\end{abstract}

\maketitle

\section{The general scope}
According to Eq. (2) of Ref. \onlinecite{CohLuoGutDobTewAla-JCP-19}, the similarity transformed Hamiltonian can be written as 
\begin{equation}
 \label{ht_def_g}
 e^{-\hat{\tau}} \hat{H} e^{\hat{\tau}} = H + \big[ H,\hat{\tau} \big] + \frac{1}{2}\bigg[ \big[H,\hat{\tau}\big],\hat{\tau}\bigg]
\end{equation}
where $\hat{\tau} = \sum_{i<j}u(\br{}_i,\br{}_j)$ and $\hat{H} = \sum_i -\frac{1}{2} \nabla^2_i + v(\br{}_i) + \sum_{i<j} \frac{1}{r_{ij}}$. 
This leads to the following similarity transformed Hamiltonian 
\begin{equation}
 \begin{aligned}
 \label{ht_def_gu}
 \tilde{H} & = H - \sum_i \bigg( \frac{1}{2} \nabla_i^2 \hat{\tau} + \big(\nabla_i \hat{\tau} \big) + \frac{1}{2} \big(\nabla_i \hat{\tau} \big)^2  \bigg) \\
           & = H - \sum_{i<j} \hat{K}(\bri{i},\bri{j}) - \sum_{i<j<k} \hat{L}(\bri{i},\bri{j},\bri{k}),
 \end{aligned}
\end{equation}
where the effective two- and three-body operators $\hat{K}(\bri{i},\bri{u})$ and $\hat{L}(\bri{i},\bri{u},\bri{k})$ are defined as
\begin{equation}
 \begin{aligned}
 \label{def_k}
  \hat{K}(\bri{1},\bri{2}) ) = \frac{1}{2} \bigg( &\nabla_1^2 u(\bri{1},\bri{2}) + \nabla_2^2u(\bri{1},\bri{2}) \\
                                               + &\big(\nabla_1 u(\bri{1},\bri{2}) \big) ^2 + \big(\nabla_1 u(\bri{1},\bri{2}) \big) ^2 \bigg) \\
                                               + &\nabla_1 u(\bri{1},\bri{2}) \cdot \nabla_1 + \nabla_2 u(\bri{1},\bri{2}) \cdot \nabla_2,
 \end{aligned}
\end{equation}
\begin{equation}
 \begin{aligned}
 \label{def_k}
  \hat{L}(\bri{1},\bri{2},\bri{3}) ) = & \nabla_1 u(\bri{1},\bri{2}) \cdot \nabla_1 u(\bri{1},\bri{3}) + \nabla_2 u(\bri{2},\bri{1}) \cdot \nabla_2 u(\bri{2},\bri{3})  \\
                                     + & \nabla_3 u(\bri{3},\bri{1}) \cdot \nabla_3 u(\bri{3},\bri{2}).
 \end{aligned}
\end{equation}

Here we propose to use a non-symmetric Jastrow factor $ u(r_{12},\mur{1})$ and therefore we need to compute the integrals of the $\hat{K}(\bri{1},\bri{2})$ and $\hat{L}(\bri{1},\bri{2},\bri{3})$ operators with such a Jastrow factor. 
Here, we will give the equations for a general $\mur{1}$. 
To derive properly the integrals for such a new Jastrow factor, one needs to derive the different terms involving gradients and so on. 


\subsection{Gradients of ${u}(r_{12},\mur{1})$}
A fundamental quantity is the gradient of ${u}(\bd{r_1},\bd{r_2})$ which is 
\begin{equation}
 \nabla_1 {u}(\bd{r_1},\bd{r_2}) = \deriv{}{x_1}{} {u}(\bd{r_1},\bd{r_2}) {\bf e}_{x_1} + \deriv{}{y_1}{} {u}(\bd{r_1},\bd{r_2}) {\bf e}_{y_1} + \deriv{}{z_1}{} {u}(\bd{r_1},\bd{r_2}) {\bf e}_{z_1}.
\end{equation}
Let us begin with the first term 
\begin{equation}
 \begin{aligned}
 \deriv{}{x_1}{} u(r_{12};\mur{1})=& \deriv{}{r_{12}}{} u(r_{12};\mur{1})\deriv{r_{12}}{x_1}{} \\ 
                                  +& \deriv{}{\mur{1}}{}u(r_{12};\mur{1}) \deriv{\mur{1}}{x_1}{},
 \end{aligned}
\end{equation}
but as 
\begin{equation}
 \deriv{}{\mu}{}u(r_{12};\mu) = \frac{e^{-(\mu r_{12})^2}}{2 \sqrt{\pi} \mu^2}
\end{equation}
one obtains 
\begin{equation}
 \begin{aligned}
  \label{eq:grad_mur_1}
 \deriv{}{x_1}{} u(r_{12};\mur{1}) =& \frac{1 - \text{erf}\big(\mu(\bri{1}) r_{12} \big)}{2 r_{12}}\big( x_1 - x_2) \\
                                   +&\frac{e^{-(\mu(\bri{1}) r_{12})^2}}{2 \sqrt{\pi} \mu(\bri{1})^2} \deriv{\mur{1}}{x_1}{}.  
 \end{aligned}
\end{equation}
Similarly, as by definition $\deriv{\mur{1}}{x_2}{} = 0$, one obtains that 
\begin{equation}
 \label{eq_grad_mur_r2}
 \deriv{}{x_2}{} u(r_{12};\mur{1}) = -\big(x_1 - x_2 \big)\frac{1 - \text{erf}(\mur{1} r_{12})}{2 r_{12}} .
\end{equation}
One can summarize the gradients of ${u}(r_{12},\mur{1})$ by the following formulas 
\begin{equation}
 \label{eq:nabla_1}
 \nabla_1 u(r_{12};\mur{1})  = {\bf u_0}(r_{12},\mur{1}) + {\bf \gamma}(\br_1, r_{12}), 
\end{equation}
with 
\begin{equation}
 \label{eq:dev_u0}
 {\bf u_0}(r_{12},\mur{1}) = u_0^x(r_{12},\mur{1}) {\bf e}_{x_1} + u_0^y(r_{12},\mur{1}) {\bf e}_{y_1} + u_0^z(r_{12},\mur{1}) {\bf e}_{z_1}, 
\end{equation}
\begin{equation}
 \label{eq:dev_u0_x}
 u_0^x(r_{12},\mur{1}) = \frac{1 - \text{erf}\big(\mu(\bri{1}) r_{12} \big)}{2 r_{12}}\big( x_1 - x_2),
\end{equation}
and
\begin{equation}
 \label{eq:dev_gamma_1}
 {\bf {\gamma}}(\br_1, r_{12}) = \frac{e^{-(\mu(\bri{1}) r_{12})^2}}{2 \sqrt{\pi} \mu(\bri{1})^2} \nabla_1{\mur{1}},
\end{equation}
and equivalently for $\nabla_2 u(r_{12};\mur{1})$  
\begin{equation}
 \label{eq:nabla_2}
 \nabla_2 u(r_{12};\mur{1}) = - {\bf u_0}(r_{12},\mur{1}).
\end{equation}

\subsection{Computation of $\bigg( \nabla_1  u(r_{12};\mur{1})\bigg)^2 + \bigg( \nabla_2  u(r_{12};\mur{1})\bigg)^2$ }
According to Eq. \eqref{eq:grad_mur_1} one has 
\begin{equation}
 \begin{aligned}
 \deriv{}{x_1}{} u(r_{12};\mur{1}) =& \frac{1 - \text{erf}\big(\mu(\bri{1}) r_{12} \big)}{2 r_{12}}\big( x_1 - x_2) \\
                                   +&\frac{e^{-(\mu(\bri{1}) r_{12})^2}}{2 \sqrt{\pi} \mu(\bri{1})^2} \deriv{\mur{1}}{x_1}{}.  
 \end{aligned}
\end{equation}
Therefore, the computation of $\big( \deriv{}{x_1}{} u(r_{12};\mur{1}) \big)^2$ yields 
\begin{equation}
 \begin{aligned}
 \bigg( \deriv{}{x_1}{} u(r_{12};\mur{1}) \bigg)^2 = & \frac{\bigg(1 - \text{erf}\big(\mu(\bri{1}) r_{12} \big)\bigg)^2}{4 \big(r_{12}\big)^2}\big( x_1 - x_2)^2 \\
 + & \frac{e^{-2(\mu(\bri{1}) r_{12})^2}}{4 \pi \mu(\bri{1})^4} \bigg(\deriv{\mur{1}}{x_1}{} \bigg)^2 \\ 
 + & \frac{1 - \text{erf}\big(\mu(\bri{1}) r_{12} \big)}{r_{12}}\big( x_1 - x_2) \frac{e^{-(\mu(\bri{1}) r_{12})^2}}{2 \sqrt{\pi} \mu(\bri{1})^2} \deriv{\mur{1}}{x_1}{}.
 \end{aligned}
\end{equation}
Therefore, the computation of $\bigg( \nabla_1  u(r_{12};\mur{1})\bigg)^2$ yields 
\begin{equation}
 \begin{aligned}
 & \bigg( \nabla_1  u(r_{12};\mur{1})\bigg)^2  = \frac{\bigg(1 - \text{erf}\big(\mu(\bri{1}) r_{12} \big)\bigg)^2}{4} \\
 & +   \frac{e^{-2(\mu(\bri{1}) r_{12})^2}}{4 \pi \mu(\bri{1})^4} \bigg( \nabla_1  \mur{1}\bigg)^2 \\ 
   + & \nabla_1  \mur{1} \, \cdot \, \big( \br{}_1 - \br{}_2\big) \frac{1 - \text{erf}\big(\mu(\bri{1}) r_{12} \big)}{r_{12}} \frac{e^{-(\mu(\bri{1}) r_{12})^2}}{2 \sqrt{\pi}\mu(\bri{1})^2} 
 \end{aligned}
\end{equation}

Eventually, the total operator yields
\begin{equation}
 \label{eq:nabl_2}
 \begin{aligned}
 & \frac{\bigg( \nabla_1  u(r_{12};\mur{1})\bigg)^2 + \bigg( \nabla_2  u(r_{12};\mur{1})\bigg)^2}{2} =  \frac{\bigg(1 - \text{erf}\big(\mu(\bri{1}) r_{12} \big)\bigg)^2}{4} \\
 &  +  \frac{1}{2}\frac{e^{-2(\mu(\bri{1}) r_{12})^2}}{4 \pi \mu(\bri{1})^4} \bigg( \nabla_1  \mur{1}\bigg)^2 \\ 
 &  +  \frac{1}{2} \nabla_1  \mur{1} \, \cdot \, \big( \br{}_1 - \br{}_2\big) \frac{1 - \text{erf}\big(\mu(\bri{1}) r_{12} \big)}{r_{12}} \frac{e^{-(\mu(\bri{1}) r_{12})^2}}{2 \sqrt{\pi}\mu(\bri{1})^2} .
 \end{aligned}
\end{equation}
Note the $\frac{1}{2}$ factor in the second and third lines of Eq. \eqref{eq:nabl_2} which is due to the fact that $\mur{1}$ depends only on $\bd{r}_1$. 
These integrals remain still possible through an analytical integration on $\bd{r}_2$ and numerical integration on $\bd{r}_1$. 

\subsection{Derivation of the Laplacian}
We need to compute the following quantity 
\begin{equation}
 \nabla_1 \cdot \nabla_1 u(r_{12};\mur{1}) + \nabla_2 \cdot \nabla_2 u(r_{12};\mur{1}).
\end{equation}
where $\nabla_1 u(r_{12};\mur{1}) $ and $\nabla_2 u(r_{12};\mur{1})$ are given by the formulas \eqref{eq:nabla_1} and \eqref{eq:nabla_2}, respectively. 
Therefore one needs to compute 
\begin{equation}
 \begin{aligned}
& \nabla_1 \cdot \nabla_1 u(r_{12};\mur{1}) + \nabla_2 \cdot \nabla_2 u(r_{12};\mur{1}) \\
=& \nabla_1 \cdot \big( {\bf u_0}(r_{12},\mur{1}) + {\bf \gamma}(\br_1, r_{12}) \big) - \nabla_2 \cdot  {\bf u_0}(r_{12},\mur{1}). 
 \end{aligned}
\end{equation}
The term in $\nabla_1 $ yields 
\begin{equation}
 \nabla_1 \cdot \big( {\bf u_0}(r_{12},\mur{1}) + {\bf \gamma}(\br_1, r_{12}) \big) = \nabla_1 \cdot  {\bf u_0}(r_{12},\mur{1}) + \nabla_1 \cdot {\bf \gamma}(\br_1, r_{12}),  
\end{equation}

Therefore, when adding  the two derivatives on has that 
\begin{equation}
 \label{eq:lapl_tot}
 \begin{aligned}
& \nabla_1 \cdot \nabla_1 u(r_{12};\mur{1}) + \nabla_2 \cdot \nabla_2 u(r_{12};\mur{1})   \\
&= \nabla_1 \cdot  {\bf u_0}(r_{12},\mur{1}) - \nabla_2 \cdot {\bf u_0}(r_{12},\mur{1}) + \nabla_1 \cdot {\bf \gamma}(\br_1, r_{12}), 
 \end{aligned}
\end{equation}

The derivative of $u_0^x(r_{12},\mur{1})$ with respect to $x_1$ yields 
\begin{equation}
 \begin{aligned}
  \deriv{}{x_1}{} u_0^x(r_{12},\mur{1}) = \deriv{}{x_1}{}\bigg|_{\mur{1}=cst} \frac{1 - \text{erf}\big(\mu(\bri{1}) r_{12} \big)}{2 r_{12}}\big( x_1 - x_2) \\ 
 + \big( x_1 - x_2\big)\deriv{}{\mur{1}}{}\frac{1 - \text{erf}\big(\mu(\bri{1}) r_{12} \big)}{2 r_{12}} \deriv{}{x_1}{} \mur{1},
 \end{aligned}
\end{equation}
and correspondingly for $x_2$ 
\begin{equation}
 \deriv{}{x_2}{} u_0^x(r_{12},\mur{1}) = \deriv{}{x_2}{}\bigg( \frac{1 - \text{erf}\big(\mu(\bri{1}) r_{12} \big)}{2          r_{12}}\big( x_1 - x_2)\bigg).
\end{equation}
One can notice that 
\begin{equation}
 \begin{aligned}
& \nabla_1\bigg|_{\mur{1}=cst} {\bf u_0}(r_{12},\mur{1})  - \nabla_2 \cdot {\bf u_0}(r_{12},\mur{1}) \\
 =& 2 \times \bigg( \frac{1 - \text{erf}(\mur{1} r_{12})}{r_{12}} - \frac{\mur{1}}{\sqrt{\pi}} e^{-\big(\mur{1} r_{12} \big)^2}  \bigg),
 \end{aligned}
\end{equation}
as it is the scalar potential obtained with the previously derived Jastrow factor but with $\mur{1}$ instead of $\mu$. 
Then, as 
\begin{equation}
 \deriv{}{\mu}{} \bigg(1 - \text{erf}\big(\mu r_{12} \big) \bigg) = - 2 \frac{r_{12}}{\sqrt{\pi}} e^{-\big( \mu r_{12}\big)^2},
\end{equation}
the second term yields 
\begin{equation}
 \begin{aligned}
 & \deriv{}{\mur{1}}{}\frac{1 - \text{erf}\big(\mu(\bri{1}) r_{12} \big)}{2 r_{12}} (\br_1 - \br_2) \cdot \nabla_1 \mur{1} \\
=& -\frac{e^{-\big( \mur{1} r_{12}\big)^2}}{\sqrt{\pi}}(\br_1 - \br_2) \cdot \nabla_1 \mur{1}. 
 \end{aligned}
\end{equation}

Eventually the computation of the Laplacian leads to 
\begin{equation}
 \begin{aligned}
 &\nabla_1 \cdot \nabla_1 u(r_{12};\mur{1}) + \nabla_2 \cdot \nabla_2 u(r_{12};\mur{1}) \\ 
 & =  2 \times \bigg( \frac{1 - \text{erf}(\mur{1} r_{12})}{r_{12}} - \frac{\mur{1}}{\sqrt{\pi}} e^{-\big(\mur{1} r_{12} \big)^2}  \bigg) \\
 & -\frac{e^{-\big( \mur{1} r_{12}\big)^2}}{\sqrt{\pi}}(\br_1 - \br_2) \cdot \nabla_1 \mur{1} + \nabla_1 \cdot {\bf \gamma}(\br_1, r_{12}).
 \end{aligned}
\end{equation}


\subsection{Derivation of the non hermitian term}
The non hermitian term can be written as 
\begin{equation}
 \begin{aligned}
&  \nabla_1 u(\bri{1},\bri{2}) \cdot \nabla_1 + \nabla_2 u(\bri{1},\bri{2}) \cdot \nabla_2 \\
& = \bigg( {\bf u_0}(r_{12},\mur{1}) + {\bf \gamma}(\br_1, r_{12}) \bigg) \cdot \nabla_1 - {\bf u_0}(r_{12},\mur{1}) \cdot \nabla_2 \\
& = {\bf u_0}(r_{12},\mur{1}) \cdot \bigg( \nabla_1 - \nabla_2 \bigg) + {\bf \gamma}(\br_1, r_{12}) \cdot \nabla_1. 
 \end{aligned}
\end{equation}
One recognizes the non hermitian term of Ref. \onlinecite{Gin-JCP-21} evaluated with a $\mur{1}$ and an additional term. 
Therefore, one can rewrite the non hermitian term as 
\begin{equation}
 \begin{aligned}
 &\nabla_1 u(\bri{1},\bri{2}) \cdot \nabla_1 + \nabla_2 u(\bri{1},\bri{2}) \cdot \nabla_2 = \\
& \bigg( 1 - \text{erf}\big(\mur{1} r_{12} \big) \bigg) \deriv{}{r_{12}}{} + {\bf \gamma}(\br_1, r_{12}) \cdot \nabla_1. 
 \end{aligned}
\end{equation}

\subsection{Sum of all terms for $\hat{K}(\bri{i},\bri{j})$ }
Adding all terms reads 
\begin{equation}
 \begin{aligned}
 &\hat{K}(\bri{i},\bri{j}) =  \frac{1 - \text{erf}(\mur{1} r_{12})}{r_{12}} - \frac{\mur{1}}{\sqrt{\pi}} e^{-\big(\mur{1} r_{12} \big)^2} \\
 & -\frac{e^{-\big( \mur{1} r_{12}\big)^2}}{2 \sqrt{\pi}}(\br_1 - \br_2) \cdot \nabla_1 \mur{1} + \frac{1}{2}\bigg(\nabla_1 \cdot {\bf \gamma}(\br_1, r_{12})\bigg) \\
 &  +  \frac{\bigg(1 - \text{erf}\big(\mu(\bri{1}) r_{12} \big)\bigg)^2}{4} \\
 &  +  \frac{1}{2}\frac{e^{-2(\mu(\bri{1}) r_{12})^2}}{4 \pi \mu(\bri{1})^4} \bigg( \nabla_1  \mur{1}\bigg)^2 \\ 
 &  +  \frac{1}{2} \nabla_1  \mur{1} \, \cdot \, \big( \br{}_1 - \br{}_2\big) \frac{1 - \text{erf}\big(\mu(\bri{1}) r_{12} \big)}{r_{12}} \frac{e^{-(\mu(\bri{1}) r_{12})^2}}{2 \sqrt{\pi}\mu(\bri{1})^2} \\ 
 &  + \bigg( 1 - \text{erf}\big(\mur{1} r_{12} \big) \bigg) \deriv{}{r_{12}}{} + {\bf \gamma}(\br_1, r_{12}) \cdot \nabla_1. 
 \end{aligned}
\end{equation}
Summing $\frac{1}{r_{12}} - \hat{K}(\bri{i},\bri{j})$ leads to 
\begin{equation}
 \begin{aligned}
  \frac{1}{r_{12}} - \hat{K}(\bri{i},\bri{j}) = & \tilde{\mathcal{W}}_{ee}(r_{ij},\mur{1}) + \tilde{t}(r_{12},\mur{1}) \\ 
                                              + & \tilde{\mathcal{U}}(r_{ij},\mur{1})
                                              +  \tilde{\mathcal{V}}(r_{ij},\mur{1})
 \end{aligned}
\end{equation}
with 
\begin{equation}
 \begin{aligned}
 \tilde{\mathcal{W}}_{ee}(r_{ij},\mur{1})  =  & \frac{\text{erf}(\mur{1} r_{ij})}{r_{ij}} 
  + \frac{\mur{1}}{\sqrt{\pi}} e^{-\big(\mur{1} r_{ij} \big)^2} \\
 & - \frac{\bigg(1 -  \text{erf}(\mur{1} r_{ij}) \bigg)^2}{4}, 
 \end{aligned}
\end{equation}
\begin{equation}
 \begin{aligned}
 \tilde{t}(r_{12},\mur{1}) =  \bigg( \text{erf}(\mur{1} r_{ij}) - 1\bigg) \deriv{}{r_{ij}}{},
 \end{aligned}
\end{equation}
\begin{equation} 
\begin{aligned}
 \tilde{\mathcal{U}}(r_{ij},\mur{1}) = & -\frac{1}{2\sqrt{\pi}}\nabla_1 \mur{1} \cdot \big(\br_1 - \br_2 \big) 
  e^{-\big( \mur{1} r_{12}\big)^2}\\ & \bigg(\frac{1 - \text{erf}\big(\mu(\bri{1}) r_{12} \big)}{2 \mu(\bri{1})^2 r_{12}} -1\bigg)
 \end{aligned}
\end{equation}
\begin{equation}
 \begin{aligned}
 \tilde{\mathcal{V}}(r_{ij},\mur{1}) = & -\frac{1}{2}\bigg( \nabla_1 \cdot \gamma(\br_1, r_{12}) \bigg) - \gamma(\br_1, r_{12}) \cdot \nabla_1 
\\ 
&- \frac{1}{2}\frac{e^{-2(\mu(\bri{1}) r_{12})^2}}{4 \pi \mu(\bri{1})^4} \bigg( \nabla_1  \mur{1}\bigg)^2
 \end{aligned}
\end{equation}

\section{Integrals computations}
\subsection{Integrals of $\tilde{\mathcal{W}}_{ee}(r_{ij},\mur{1})$ }
One needs to compute the integrals of that type
\begin{equation}
 \label{w_ee_ijkl}
 \begin{aligned}
 &\matelem{kl}{\tilde{\mathcal{W}}_{ee}(r_{ij},\mur{1})}{ij} \\ & = \int \text{d}\br_{1} \text{d}\br_{2} \phi_i(\br_1) \phi_k(\br_1) \tilde{\mathcal{W}}_{ee}(r_{ij},\mur{1}) \phi_i(\br_2) \phi_l(\br_2) \\
 &\matelem{kl}{\frac{\text{erf}(\mur{1} r_{ij})}{r_{ij}}}{ij} 
 + \matelem{kl}{\frac{\mur{1}}{\sqrt{\pi}} e^{-\big(\mur{1} r_{ij} \big)^2}}{ij} \\ 
& - \frac{1}{4}\matelem{kl}{\bigg(1 -  \text{erf}(\mur{1} r_{ij}) \bigg)^2}{ij}.
 \end{aligned}
\end{equation}
By defining the following integrals
\begin{equation}
 w_{jl}(\br,\mu) = \int \text{d}\br' \phi_j(\br') \phi_l(\br') \frac{\text{erf}(\mu |\br - \br' | )}{ |\br - \br' |},
\end{equation}
\begin{equation}
 G_{jl}(\br,\alpha) = \int \text{d}\br' \phi_j(\br') \phi_l(\br') e^{-\big(\alpha |\br - \br' |\big)^2 } ,
\end{equation}
one can compute the two first integrals in Eq. \eqref{w_ee_ijkl} as 
\begin{equation}
 \matelem{kl}{\frac{\text{erf}(\mur{1} r_{ij})}{r_{ij}}}{ij} = \int \text{d}\br \phi_i(\br) \phi_k(\br) w_{jl}(\br,\mur{}), 
\end{equation}
\begin{equation}
 \matelem{kl}{\frac{\mur{1}}{\sqrt{\pi}} e^{-\big(\mur{1} r_{ij} \big)^2}}{ij} = \int \text{d}\br \phi_i(\br) \phi_k(\br)\frac{\mur{}}{\sqrt{\pi}} G_{jl}(\br,(\mur{})^2). 
\end{equation}

The last term $\matelem{kl}{\bigg(1 -  \text{erf}(\mur{1} r_{ij}) \bigg)^2}{ij}$ can be computed as
\begin{equation}
 \label{eq:k_ijkl}
 \begin{aligned}
& \matelem{kl}{\bigg(1 -  \text{erf}(\mur{1} r_{ij}) \bigg)^2}{ij} \\ & = \int \dr{1} \dr{2} \phi_i(\bri{1}) \phi_j(\bri{2}) \big( g(r_{12},\mur{1}) \big)^2  \phi_k(\bri{1}) \phi_l(\bri{2}),  
 \end{aligned}
\end{equation}
with 
\begin{equation}
 g(x, \mu)= 1 - \text{erf}( \mu x).
\end{equation}
To make integrals analytical, we first fit the function $\text{erfc}(x)$ with a simple Slater-Gaussian function 
\begin{equation}
 \text{erfc}(x) \approx h(x,\alpha,\beta,c)
\end{equation}
with 
\begin{equation}
 h(x,\alpha,\beta,c) = e^{-\alpha x - \beta x^2}
\end{equation}
and $\alpha=1.09529$ and $\beta = 0.756023$. 
Then, by posing $y=\mu x$, one obtains 
\begin{equation}
 \label{fit_erf}
 \begin{aligned}
  g(x,\mu)  \approx & e^{-\alpha \mu x - \beta (\mu x)^2}\\ 
        =& h(x,\alpha \mu, \beta \mu^2).
 \end{aligned}
\end{equation}
Therefore, one can fit $g(x)^2$ as 
\begin{equation}
 \begin{aligned}
 g(x,\mu)^2&= \bigg( 1 - \text{erf}(\mu x) \bigg)^2\\
           &= \bigg( e^{-\alpha \mu x } e^{-\beta \mu^2 x^2}\bigg)^2 \\
           &= e^{-2\alpha  \mu x } e^{-2 \beta \mu^2 x^2} \\
           &= h(x,2 \alpha \mu, 2 \beta \mu^2).
 \end{aligned}
\end{equation}
Then we fit the Slater function as a linear combination of Gaussians 
\begin{equation}
 e^{-X} = \sum_{m=1}^{N_s} c_m e^{-\zeta_m X^2}. 
\end{equation}
In the present work, we use $N_s=20$ and the $\{c_m,\zeta_m\}$ parameters are reported in Table \ref{gauss_fit}.
\begin{table}
\label{gauss_fit}
\caption{Set of coefficients $c_m$ and exponents $\zeta_m$ for the fit of $e^{-X}$}
\begin{ruledtabular}
\begin{tabular}{ll}
 $\zeta_m$ & $c_m$ \\
\hline                 
   30573.77073         & 0.00338925525  \\
   5608.452381         & 0.00536433869  \\
   1570.956734         & 0.00818702846  \\
   541.3978511         & 0.01202047655  \\
   212.4346963         & 0.01711289568  \\
   91.31444574         & 0.02376001022  \\
   42.04087246         & 0.03229121736  \\
   20.43200443         & 0.04303646818  \\
   10.37775161         & 0.05624657578  \\
   5.468807545         & 0.07192311571  \\
   2.973735292         & 0.08949389001  \\
   1.661441902         & 0.10727599240  \\
   0.9505256082        & 0.12178961750  \\
   0.5552868397        & 0.12740141870  \\
   0.3304336002        & 0.11759168160  \\
   0.1998230323        & 0.08953504394  \\
   0.1224684076        & 0.05066721317  \\
   0.07575825322       & 0.01806363869  \\
   0.04690146243       & 0.00305632563  \\
   0.02834749861       & 0.00013317513  \\
\end{tabular}
\end{ruledtabular}
\end{table}
By posing $X=\gamma x$ one can fit any Slater function as
\begin{equation}
 e^{-\gamma x} = \sum_{m=1}^{N_s} c_m e^{-\zeta_m \gamma^2 x^2}. 
\end{equation}
Eventually, the function $g(x,\mu)^2$ is obtained as a linear combination of Gaussian
\begin{equation}
 g(x,\mu)^2 \approx \sum_{m=1}^{N_s} c_m e^{-2\mu^2\big(2 \alpha \zeta_m + \beta\big) x^2},
\end{equation}
which makes then the integrals analytical
\begin{equation}
 \begin{aligned}
 \matelem{kl}{\bigg(1 -  \text{erf}(\mur{1} r_{ij}) \bigg)^2}{ij}  \approx \sum_{m=1}^{N_s} c_m &\int \dr{1} \dr{2}  \phi_i(\bri{1}) \phi_j(\bri{2}) \\ & \phi_k(\bri{1}) \phi_l(\bri{2}) e^{-2\mur{}^2\big(2 \alpha \zeta_m + \beta\big) (r_{12})^2} ,
 \end{aligned}
\end{equation}
which then can be computed as
\begin{equation}
 \begin{aligned}
&  \matelem{kl}{\bigg(1 -  \text{erf}(\mur{1} r_{ij}) \bigg)^2}{ij}  \\ & \approx \int \dr{} \phi_i(\br{}) \phi_k(\br{}) \sum_{m=1}^{N_s} c_m G_{jl}(\br,2\mur{}^2\big(2 \alpha \zeta_m + \beta\big)).
 \end{aligned}
\end{equation}
All numerical tests performed for $\mu > 0.1$ show that this fit is highly accurate. 

\subsection{Integrals of $\tilde{t}(r_{12},\mur{1})$}
A more practical point of view for the computation of the non hermitian term is the following 
\begin{equation}
 \begin{aligned}
 \tilde{t}(r_{12},\mur{1}) =& \frac{\text{erf}(\mur{} r_{12})-1}{2 r_{12}} \\ 
                           &\bigg( (x_1 - x_2) \deriv{}{x_1}{} + (y_1 - y_2) \deriv{}{y_1}{} + (z_1 - z_2) \deriv{}{z_1}{}\bigg),
 \end{aligned}
\end{equation}
\begin{equation}
 \begin{aligned}
 \label{def_non_hermit}
& \tilde{t}(r_{12},\mur{1})  = \frac{\text{erf}(\mur{} r_{12})-1}{2 r_{12}} \\
& \bigg( (x_1 - x_2) \big( \deriv{}{x_1}{} - \deriv{}{x_2}{} \big) +
         (y_1 - y_2) \big( \deriv{}{y_1}{} - \deriv{}{y_2}{} \big)  \\
&  +      (z_1 - z_2) \big( \deriv{}{z_1}{} - \deriv{}{z_2}{} \big)\bigg).                                                
 \end{aligned}
\end{equation}
We define the following integrals
\begin{equation}
 {w}_{jl}^{x}(\br,\mu) = \int \dr{}' \phi_l(\br') x' \frac{\text{erf}(\mu |\br' - \br| )-1}{|\br' - \br|} \phi_j(\br'),
\end{equation}
\begin{equation}
 {w'}_{jl}^{x}(\br,\mu) = \int \dr{}' \phi_l(\br') \frac{\text{erf}(\mu |\br' - \br| )-1}{|\br' - \br|}\deriv{}{x'}{} \phi_j(\br'),
\end{equation}
\begin{equation}
 {w'}_{jl}^{xx}(\br,\mu) = \int \dr{}' x' \, \,\phi_l(\br') \frac{\text{erf}(\mu |\br' - \br| )-1}{|\br' - \br|}\deriv{}{x'}{} \phi_j(\br'),
\end{equation}
and we decompose the integral $\matelem{kl}{\tilde{t}(r_{12},\mur{1})}{ij}$ in $x,y,z$ components 
\begin{equation}
 \begin{aligned}
  \matelem{kl}{\tilde{t}(r_{12},\mur{1})}{ij} =& \matelem{kl}{\tilde{t}_x(r_{12},\mur{1})}{ij} 
                                              + \matelem{kl}{\tilde{t}_y(r_{12},\mur{1})}{ij} \\
                                              +& \matelem{kl}{\tilde{t}_z(r_{12},\mur{1})}{ij},
 \end{aligned}
\end{equation}
with 
\begin{equation}
 \begin{aligned}
  \matelem{kl}{\tilde{t}_x(r_{12},\mur{1})}{ij} = & \int \dr{1} \dr{2} \phi_l(\br_2) \phi_k(\br_1) \frac{\text{erf}(\mur{} r_{12}) - 1}{2 r_{12}} \\
& (x_1 - x_2) \big( \deriv{}{x_1}{} - \deriv{}{x_2}{} \big) \phi_i(\br_1) \phi_j(\br_2).
 \end{aligned}
\end{equation}
The integral $\matelem{kl}{\tilde{t}_x(r_{12},\mur{1})}{ij}$ can then be computed as 
\begin{equation}
 \begin{aligned}
 \matelem{kl}{\tilde{t}_x(r_{12},\mur{1})}{ij} \\
  =& \int \dr{} \phi_k(\br) x \deriv{}{x}{} \phi_i(\br) \frac{1}{2}\big(w_{jl}(\br,\infty) - w_{jl}(\br,\mur{})\big) \\
  +& \int \dr{} \phi_k(\br) \phi_i(\br) \frac{1}{2}\big({w'}_{jl}^{xx}(\br,\infty) - {w'}_{jl}^{xx}(\br,\mur{})\big) \\
  -& \int \dr{} \phi_k(\br) x \phi_i(x) \frac{1}{2} \big( {w'}_{jl}^x(\br,\infty) - {w'}_{jl}^x(\br,\mur{}) \big) \\
  -& \int \dr{} \phi_k(\br) \deriv{}{x}{} \phi_i(\br) \frac{1}{2} \big( {w}_{jl}^x(\br,\infty) - {w}_{jl}^x(\br,\mur{}) \big)
 \end{aligned}
\end{equation}

\bibliography{srDFT_SC}


\end{document}
