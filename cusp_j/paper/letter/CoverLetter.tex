\documentclass[10pt]{letter}
\usepackage{UPS_letterhead,color}
%\usepackage{xcolor,mhchem,mathpazo,ragged2e}
\newcommand{\alert}[1]{\textcolor{red}{#1}}


\begin{document}

\begin{letter}%
{To the Editors of the Journal of Chemical Physics}

\opening{Dear Editors,}

%\justifying
Please find enclosed my manuscript entitled
\begin{quote}
\textit{``A new form of transcorrelated Hamiltonian inspired by range-separated DFT''}, 
\end{quote}
which I would like you to consider as a full article in the \textit{Journal of Chemical Physics}.
%This contribution fits nicely in the section \textit{``Spectroscopy and Photochemistry; General theory''}. 

One of the most fundamental drawbacks of conventional wave function methods is the slow convergence of energies and properties with respect to the one-electron basis set.
As proposed by Kutzelnigg more than thirty years ago, one can introduce a Jastrow factor explicitly depending on the interelectronic distance $r_{12}$ to significantly speed up the convergence. 
In this framework, one can take into account the effect of such Jastrow factor directly into an effective Hamiltonian: the transcorrelated Hamiltonian. The generic form of such an effective Hamiltonian have been established in the late 60's by Boys and Handy, but a renewal of interest have emerged recently due to the fast convergence of its eigenvalues with respect to the basis set. 

In the present work, we propose the analytical derivation and first numerical test of a new universal Jastrow factor specially designed for the transcorrelated framework. The equations leading to this new Jastrow factor are inspired by the range-separated DFT effective Hamiltonian. Such a Jastrow factor has several advantages: it is tuned by a unique parameter allowing to simply adapt to a typical scale-length, the explicit analytical form of the transcorrelated Hamiltonian is directly known and its eigenvalues converge extremely fast toward the complete basis set limit, all two-electron integrals can be easily performed analytically, the expensive three-body terms can be computed efficiently through a mixed numerical-analytical approach.  

Because of the large impact of our work, we expect it to be of interest to a wide audience within the theoretical chemistry community and beyond.
We suggest Seiichiro L. Ten-No, Ali Alavi, David Tew and Naoto Umezawa as potential referees.
This contribution has never been submitted in total nor in parts to any other journal. 
We look forward to hearing from you soon.

\closing{Sincerely, the author.}


\end{letter}
\end{document}






